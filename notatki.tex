\documentclass{article}
\usepackage[utf8]{inputenc}
\usepackage{amsfonts}
\usepackage{soul}
\usepackage{textcomp}
\usepackage{color}
\usepackage{enumitem}
\usepackage{amsmath}
\usepackage{amssymb}
\usepackage{mathtools}
\usepackage{listings}
\usepackage{amsthm}
\definecolor{dkgreen}{rgb}{0,0.6,0}
\definecolor{gray}{rgb}{0.5,0.5,0.5}
\definecolor{mauve}{rgb}{0.58,0,0.82}
\lstset{frame=tb,
    language=Bash,
    aboveskip=3mm,
    belowskip=3mm,
    showstringspaces=false,
    columns=flexible,
    basicstyle={\small\ttfamily},
    numbers=none,
    numberstyle=\tiny\color{gray},
    keywordstyle=\color{blue},
    commentstyle=\color{dkgreen},
    stringstyle=\color{mauve},
    breaklines=true,
    breakatwhitespace=true,
    escapeinside={(*}{*)},          % if you want to add LaTeX within your code
    tabsize=4
}

\title{Notatki z Logiki i teorii typów}
\author{Jacek Olczyk}
\date{Feb 2019}

\begin{document}
\maketitle

Check if Peirce's law: $((p \rightarrow q) \rightarrow p) \rightarrow p$ is a tautology.
$$\begin{array}{c|c|c|c|c}
p&q&p \rightarrow q&(p \rightarrow q) \rightarrow p&all\\
\hline
0&0&1&0&1\\
0&1&1&0&1\\
1&0&0&1&1\\
1&1&1&1&1
\end{array}$$
Let $ v: $ Prop Variables $ \rightarrow \{0,1\}$. Then we can define semantics of formulas $ [[\alpha]]_v $ given a valuation $ v $. What if $ v: $ Prop Variables $ \rightarrow \mathbb{P}(\mathbb{R}^2)$?
\begin{itemize}
	\item $ [[p]]_v=v(p) $
	\item $ [[\alpha\wedge\beta]]_v=[[\alpha]]_v\cup[[\beta]]_v $
	\item The same for disjunction.
	\item $ [[\neg\alpha]]_v=\setminus[[\alpha]]_v $
	\item $ [[\alpha \rightarrow \beta]]_v=\setminus[[\alpha]]_v\cup[[\beta]]_v $
\end{itemize}
Now the formula $ \alpha $ is a tautology iff for all $ v: $ Prop Var $  \rightarrow \mathbb{P}(\mathbb{R}^2) $ we have $ [[\alpha]]_v=\mathbb{R}^2 $. Prove that.\\
Boolean algebra - $ <B, \le, \cup, \cap, -, 0, 1> $\\
\section{Lambda calculus}
Normalize.
\begin{align*}
(\lambda x.x(x(y z))x)(\lambda u.uv) &= (\lambda u.uv)((\lambda u.uv)(yz))(\lambda u.uv)\\
&=(\lambda u.uv)(yzv)(\lambda u.uv)\\
&=(yzvv)(\lambda u.uv)
\end{align*}
\begin{align*}
H&=\lambda xy.x(\lambda z.yzy)x\\
HIH &= I (\lambda z. HzH)I\\
&=(\lambda z. HzH)I\\
&=HIH
\end{align*}
Homework: for every $n$, design a term that will reduce to itself in exactly $ n $ reduction.
\begin{align*}
L&=\lambda xy.x(yy)x
LLI&=
\end{align*}
This produces an infinite chain of reductions that never repeat:
$ \omega_3 =\lambda x.xxx$
\subsection{Church numerals}
Sidenote: $ <M, N> = \lambda x .xMN$
\begin{align*}
\underline{0}&=\lambda fx.x\\
\underline{1}&=\lambda fx.fx\\
\underline{2}&=\lambda fx.f(fx)\\
\underline{1}&=\lambda fx.f^{(n)}(x)\\
\text{Succ } n &=\lambda kfx.f(kf(x))\\
\text{Add } n m&=\lambda nm. n \text{ succ }m\\
\text{Multiply } n m&=\lambda nm. n (\text{add } m) \underline{0}\\
\text{Pred } n&=\pi_1(n\text{ Step }<0,0>)\\
&\text{ where Step } p= <\pi_2p, \text{ succ }(\pi_2p)>
\end{align*}

\end{document}